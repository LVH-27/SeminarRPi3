\documentclass[12pt,a4paper]{article}

\usepackage[croatian]{babel}
\usepackage[utf8]{inputenc}
\usepackage[hyphens]{url}


\usepackage[margin=2cm]{geometry}
\usepackage[colorlinks=true,urlcolor=black]{hyperref}
\pagenumbering{gobble}

\begin{document}
	\title{Očitavanje senzorskih podataka korištenjem računala Raspberry Pi 3}


	\date{\vspace{-5ex} 14.03.2017.}
	\maketitle

Ovaj će seminarski rad obraditi problematiku korištenja računala Raspberry Pi 3 za prikupljanje podataka sa senzora. Ukratko će se opisati sklopovska arhitektura računala Raspberry Pi 3 te pripadajuća programska podrška, uz nekoliko primjera korištenja. \\ \par
Kao vrlo pristupačno malo računalo, Raspberry Pi je vrlo popularan kao ugradbeno računalo, a budući da ugradbena računala vrlo često za svoj rad koriste raznolike senzore, za ovaj će rad biti ključno razumjeti osnovne principe povezivanja senzora i računala, uzimajući u obzir i sklopovski i programski aspekt. Shodno tome, bit će izložen kratak opis nekoliko akcelerometara, u funkciji senzora za mjerenje vibracija, i mikrofona, poglavito u funkciji senzora glasnoće. \\ \par
Ukratko će se opisati neke od dostupnih biblioteka i programskih okvira namijenjenih za rad sa senzorima, s naglaskom na već spomenute akcelerometre i mikrofone. Konačno, bit će pokazan i jednostavan primjer programskog koda za očitavanje podataka sa senzora uz prateći primjer vizualnog prikaza senzorskih podataka. \\ \par

\begin{enumerate}
	\setlength{\itemsep}{1pt}
	\setlength{\parskip}{0pt}
	\setlength{\parsep}{0pt}

	\item \textbf{\underline{Uvod}}
	\item \textbf{\textbf{\underline{Raspberry Pi 3}}}
	\begin{enumerate}
		\item \textbf{Sklopovlje}
		\item \textbf{Programska podrška}
		\item \textbf{Primjeri korištenja}
	\end{enumerate}
	\item \textbf{\underline{Senzori}}
	\begin{enumerate}
		\item \textbf{Akcelerometri}
		\begin{enumerate}
			\item ADXL345
			\item MMA8451
			\item LIS3DH
		\end{enumerate}
		\item \textbf{Mikrofoni}
		\begin{enumerate}
			\item placeholder 1
			\item placeholder 2
		\end{enumerate}
	\end{enumerate}
	\item \textbf{\textbf{\underline{Pregled dostupnih programskih okvira}}}
	\item \textbf{\underline{Primjeri}}
	\begin{enumerate}
		\item \textbf{Programski kôd}
		\item \textbf{Pročitani podaci}
	\end{enumerate}
	\item \textbf{\underline{Zaključak}}
	\item \textbf{\underline{Literatura}} \\
\end{enumerate}

\textbf{Literatura:}
\begin{itemize}
	\item Raspberry Pi službene stranice: \url{https://www.raspberrypi.org/}
	\item ADXL345: \url{https://learn.adafruit.com/adxl345-digital-accelerometer}
	\item MMA8451: \url{https://learn.adafruit.com/adafruit-mma8451-accelerometer-breakout/wiring-and-test?view=all}
	\item LIS3DH: \url{https://learn.adafruit.com/adafruit-lis3dh-triple-axis-accelerometer-breakout/downloads?view=all}
	\item Zvučni senzori: \url{https://www.sunfounder.com/learn/sensor-kit-v2-0-for-raspberry-pi-b-plus/lesson-19-sound-sensor-sensor-kit-v2-0-for-b-plus.html}
\end{itemize}
\end{document}